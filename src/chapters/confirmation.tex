# Introduction {#literature-short} 


The detection of high mass black holes ($>100$ M${}_\odot$) will shed light on the formation of globular clusters,
supermassive black holes and thus galaxy formation [@lodato2006supermassive; @2018IMBHreview]. LIGO is theoretically
sensitive to the merger of binary black holes with total masses up to 500 M${}_\odot$ which are expected to occur at a
rate of 0-10 yr$^{-1}$ [@fregeau2006imbhbRatePrediction, @mandel2008rates]. However, even after @salemi2019search's
targeted match-filter based search for gravitational waves from high-mass black holes the largest total mass detected so
far is approximately 80 M${}_\odot$ [@abbott2019gwtc]. A possible explanation for the absence of high mass events may be
due to their misclassification as short-duration instrumental noise transients [@blipGlitches]. High-mass mergers have
very few in-band wave cycles, and hence can easily be mistaken for short-duration instrumental transients. More details
on the theorised formation channels and merger rates are discussed in Appendix \@ref(literature).

The focus of this PhD is on improving our scientific understanding of these high-mass binary black hole systems. We
start by developing a targeted search for gravitational waves from high-mass black hole systems. This new targeted
search utilises Bayesian inference and thus is a more sensitive tool for detecting high-mass binary black hole mergers
as compared to traditional match-filtering searches. We have applied this technique on the high-mass triggers during
LIGO's second observing run to investigate the possibility of discovering new gravitational-wave signals from an
entirely new class of high mass black hole binaries.




# Significance of a high-mass Bayesian Search {#significance}

There are two primary motivations for a targeted high-mass search that utilises Bayesian inference. The first is to help
distinguish gravitataional-waves from high-mass systems from noise-transients, and the second is to try to alleviate the
discord in the significance of some of the "quiet" gravitational wave events between various search pipelines. 

## A better Detection-statistic

### Gravitational-wave triggers and short duration glitches
A foreground trigger is a section of LIGO data identified by data searching pipelines to potentially harbour some
gravitational wave signal (more on triggers in Appendix~\@ref(searches)). In the ground-based gravitational wave
detectors, many spurious instrumental noise events fool the pipelines into considering the noise events as triggers
(false-positive triggers). These instrumental-triggers are called glitches. Luckily, unlike triggers from astrophysical
sources, glitches do not often occur at the same time or with the same morphology across the various gravitational wave
detectors -- they are often _incoherent_ amongst detectors. Additionally, these short-duration glitches appear very
similar to high-mass gravitational-wave signals, making the two challenging to distinguish. For example, consider
Figure~\@ref(fig:compareGlitchAndSignal) that compares a plot of a glitch on the left and a plot of a high-mass GW
candidate on the right.


(ref:compareGlitchAndSignal) Spectrogram of a glitch on the left, and a candidate GW trigger on the right.

```{r compareGlitchAndSignal, echo=FALSE, fig.show='hold', fig.asp=.7, fig.width=6, fig.align='center', out.width='45%', fig.cap='(ref:compareGlitchAndSignal)', fig.scap="Comparing glitches to high-mass candidates."  }
knitr::include_graphics(c("images/glitch.png", "images/L1_spectrogram_logy_0.png"))
```


### Assigning significance to Gravitational-wave triggers
The LIGO-Virgo collaboration uses various methods to avoid marking false-positive glitch triggers as gravitational wave
triggers. One method first involves generating a set of _background triggers_, false-positives picked up from the search
through artificially^[The artificial gravitational wave data
should have no true gravitational waves as it is constructed by applying relative time offsets between the data of different detectors (time offsets longer than the light-travel time between the detectors).] created gravitational
wave data containing no signals. Then, a detection-statistic value, a value
that ranks the likelihood of a trigger originating from a gravitational-wave signal,  is calculated for both the
foreground and background triggers. Finally, the background trigger detection-statistic distribution can assign a probability to the _null hypothesis_ that the foreground triggers are consistent with the background. This null hypothesis probability is obtained by first calculating the cumulative distribution function for the background distribution. Placing the foreground event on the background cumulative distribution function provides the  _p-value_, or false alarm probability, of the foreground event. This p-value is a measure of how significant indicating the foreground event is from the background distribution of false-positives. 

For example, consider the
detection-statistic plot for GW150914, as seen in Figure~\@ref(fig:GW150914background). If the
foreground triggers are far away from the background trigger distribution, the foreground triggers have low p-values (and are significantly
different from the background). Hence, the low p-values foreground triggers might originate from astrophysical sources. On the other hand, if the foreground
triggers fall along with the background distribution with high p-values, the foreground triggers are consistent with the background. Thus,
they are less likely to originate from astrophysical sources.  

(ref:GW150914background) A binary coalescence search displaying a number of candidate GW events (orange markers) and the mean number of background events (black line), taken from @abbott2016observation.

```{r GW150914background, echo=FALSE,fig.pos = "!h", fig.asp=.7, fig.width=6, fig.align='center',  out.width='75%', fig.cap='(ref:GW150914background)', fig.scap="Ranking statistic plot for GW150914"}
knitr::include_graphics("images/GW150914_background.png")
```

### A Bayesian Detection statistic

Comparing foreground events to a background distribution is a "frequentist" method of assigning significance. It does
not use the literal meaning of the likelihood -- which is a probability -- or compute the posterior probability of there
being a signal given data and hypotheses. By using a frequentist method of assigning significance, these searches
discard the physical meaning of the likelihood. In principle, a Bayesian approach would be able to incorporate this
information that is lost.

The goal of this PhD's first project is to enhance the significance of high-mass foreground triggers using a Bayesian
approach and thereby increase the confidence in the detection of gravitational waves from large-mass compact binaries.
If this work does improve the significance of high-mass triggers, we will increase the likelihood of detecting a
high-mass black hole system via gravitational waves. The detection statistic we will utilise is the Bayesian Coherence
Ratio (or BCR), as seen in @bcr_paper. The BCR is discussed in detail in Appendix~\@ref(BayesianCoherenceRatio). In
short, The `BCR` is an odds (evidence ratio) comparing two hypotheses: there's a coherent GW in the data, plus Gaussian
noise vs there's only incoherent Gaussian noise or a glitch (modeled as a compact binary coalescence) in each detector
[@bcr_paper].




## A better measure of significance

### Gravitational wave candidates with uncertain significance
In aLIGO's first two observing runs, eleven gravitational wave events were found in the data by the LIGO-Virgo
scientific collaboration [@abbott2019gwtc]. Since the public release of LIGO's first and second observing run's data,
several groups have searched the data for gravitational waves independently of LIGO. One particular research group of
interest is a research team at the Institute for Advanced Study (IAS). The group constructed searches to look for the
LIGO-confirmed the gravitational wave events detected by the LIGO-Virgo collaboration, and in the process of doing so,
claim to have discovered several others events [@IAS0; @IAS1; @IAS2]. Some of these events have total masses $>85
M_{\odot}$, which is larger than the average total mass of the LIGO detections. Some of these IAS and LIGO events are
displayed in Table~\@ref(tab:O2significancesWObcr) with their p-astro reported by various LIGO and IAS search pipelines.

```{r O2significancesWObcr, tidy=FALSE, echo=FALSE}
data <- read.csv('data/significances_without_bcr.csv',  stringsAsFactors = FALSE)
knitr::kable(
  data, booktabs = TRUE, escape =F, 
  caption = "p-astro from seveal detection pipelines for a subset of the O2 foreground triggers.",
  caption.short = "p-astro for various O2 foreground triggers" 
)
```

From this Table~\@ref(tab:O2significancesWObcr), it is evident there is some uncertainty if these events can be considered real
gravitational-wave events -- are these events significantly different from the background or not? The various pipelines
have different answers. Our Bayesian search pipeline, as described in the next section, can help us answer this question 
and come to a conclusion about the significance of the events.

### Using the BCR to calculate significance
While the BCR might not increase the significance of some of the gravitational wave candidates, it will provide a more
accurate measure of their significance. The BCR significance will be more accurate as it measures the probability
that signals in two or more detectors are _coherent_ (signals share exactly the same source properties in the different
detectors) vs the detectors having _incoherent_ glitches/noise. This Bayesian significance might alleviate the discord
between the significances of specific foreground triggers reported by various search pipelines.

In summary, we will use the BCR to analyse high-mass foreground triggers to help discern signals from glitches and
provide a better measure of significance for the foreground triggers. For our work, we will initially focus on
conducting this BCR search on O2 data, and subsequently with a search in O3 data.


