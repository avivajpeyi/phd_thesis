# Introduction {#literature-short} 


The detection of high mass black holes ($>100$ M${}_\odot$) will shed light on the formation of globular clusters,
supermassive black holes and thus galaxy formation [@lodato2006supermassive; @2018IMBHreview]. LIGO is theoretically
sensitive to the merger of binary black holes with total masses up to 500 M${}_\odot$ which are expected to occur at a
rate of 0-10 yr$^{-1}$ [@fregeau2006imbhbRatePrediction, @mandel2008rates]. However, even after @salemi2019search's
targeted match-filter based search for gravitational waves from high-mass black holes the largest total mass detected so
far is approximately 80 M${}_\odot$ [@abbott2019gwtc]. A possible explanation for the absence of high mass events may be
due to their misclassification as short-duration instrumental noise transients [@blipGlitches]. High-mass mergers have
very few in-band wave cycles, and hence can easily be mistaken for short-duration instrumental transients. More details
on the theorised formation channels and merger rates are discussed in Appendix \@ref(literature).

The focus of this PhD is on improving our scientific understanding of these high-mass binary black hole systems. We
start by developing a targeted search for gravitational waves from high-mass black hole systems. This new targeted
search utilises Bayesian inference and thus is a more sensitive tool for detecting high-mass binary black hole mergers
as compared to traditional match-filtering searches. We have applied this technique on the high-mass triggers during
LIGO's second observing run to investigate the possibility of discovering new gravitational-wave signals from an
entirely new class of high mass black hole binaries.




# Significance of a high-mass Bayesian Search {#significance}

There are two primary motivations for a targeted high-mass search that utilises Bayesian inference. The first is to help
distinguish gravitataional-waves from high-mass systems from noise-transients, and the second is to try to alleviate the
discord in the significance of some of the "quiet" gravitational wave events between various search pipelines. 

## A better Detection-statistic

### Gravitational-wave triggers and short duration glitches
A foreground trigger is a section of LIGO data identified by data searching pipelines to potentially harbour some
gravitational wave signal (more on triggers in Appendix~\@ref(searches)). In the ground-based gravitational wave
detectors, many spurious instrumental noise events fool the pipelines into considering the noise events as triggers
(false-positive triggers). These instrumental-triggers are called glitches. Luckily, unlike triggers from astrophysical
sources, glitches do not often occur at the same time or with the same morphology across the various gravitational wave
detectors -- they are often _incoherent_ amongst detectors. Additionally, these short-duration glitches appear very
similar to high-mass gravitational-wave signals, making the two challenging to distinguish. For example, consider
Figure~\@ref(fig:compareGlitchAndSignal) that compares a plot of a glitch on the left and a plot of a high-mass GW
candidate on the right.


(ref:compareGlitchAndSignal) Spectrogram of a glitch on the left, and a candidate GW trigger on the right.

```{r compareGlitchAndSignal, echo=FALSE, fig.show='hold', fig.asp=.7, fig.width=6, fig.align='center', out.width='45%', fig.cap='(ref:compareGlitchAndSignal)', fig.scap="Comparing glitches to high-mass candidates."  }
knitr::include_graphics(c("images/glitch.png", "images/L1_spectrogram_logy_0.png"))
```


### Assigning significance to Gravitational-wave triggers
The LIGO-Virgo collaboration uses various methods to avoid marking false-positive glitch triggers as gravitational wave
triggers. One method first involves generating a set of _background triggers_, false-positives picked up from the search
through artificially^[The artificial gravitational wave data
should have no true gravitational waves as it is constructed by applying relative time offsets between the data of different detectors (time offsets longer than the light-travel time between the detectors).] created gravitational
wave data containing no signals. Then, a detection-statistic value, a value
that ranks the likelihood of a trigger to originate from a gravitational-wave signal,  is calculated for both the
foreground and background triggers. Finally, the background trigger detection-statistic distribution can assign a probability to the _null hypothesis_ that the foreground triggers are consistent with the background. This null hypothesis probability is obtained by first calculating the cumulative distribution function for the background distribution. Placing the foreground event on the background cumulative distribution function provides the  _p-value_, or false alarm probability, of the foreground event. This p-value is a measure of how significant indicating the foreground event is from the background distribution of false-positives. 

For example, consider the
detection-statistic plot for GW150914, as seen in Figure~\@ref(fig:GW150914background). If the
foreground triggers are far away from the background trigger distribution, the foreground triggers have low p-values (and are significantly
different from the background). Hence, the low p-values foreground triggers might originate from astrophysical sources. On the other hand, if the foreground
triggers fall along with the background distribution with high p-values, the foreground triggers are consistent with the background. Thus,
they are less likely to originate from astrophysical sources.  

(ref:GW150914background) A binary coalescence search displaying a number of candidate GW events (orange markers) and the mean number of background events (black line), taken from @abbott2016observation.

```{r GW150914background, echo=FALSE,fig.pos = "!h", fig.asp=.7, fig.width=6, fig.align='center',  out.width='75%', fig.cap='(ref:GW150914background)', fig.scap="Ranking statistic plot for GW150914"}
knitr::include_graphics("images/GW150914_background.png")
```

### A Bayesian Detection statistic

Comparing foreground events to a background distribution is a "frequentist" method of assigning significance. It does
not use the literal meaning of the likelihood -- which is a probability -- or compute the posterior probability of there
being a signal given data and hypotheses. By using a frequentist method of assigning significance, these searches
discard the physical meaning of the likelihood. In principle, a Bayesian approach would be able to incorporate this
information that is lost.

The goal of this PhD's first project is to enhance the significance of high-mass foreground triggers using a Bayesian
approach and thereby increase the confidence in the detection of gravitational waves from large-mass compact binaries.
If this work does improve the significance of high-mass triggers, we will increase the likelihood of detecting a
high-mass black hole system via gravitational waves. The detection statistic we will utilise is the Bayesian Coherence
Ratio (or BCR), as seen in @bcr_paper. The BCR is discussed in detail in Appendix~\@ref(BayesianCoherenceRatio). In
short, The `BCR` is an odds (evidence ratio) comparing two hypotheses: there's a coherent GW in the data, plus Gaussian
noise vs there's only incoherent Gaussian noise or a glitch (modeled as a compact binary coalescence) in each detector
[@bcr_paper].




## A better measure of significance

### Gravitational wave candidates with uncertain significance
In aLIGO's first two observing runs, eleven gravitational wave events were found in the data by the LIGO-Virgo
scientific collaboration [@abbott2019gwtc]. Since the public release of LIGO's first and second observing run's data,
several groups have searched the data for gravitational waves independently of LIGO. One particular research group of
interest is a research team at the Institute for Advanced Study (IAS). The group constructed searches to look for the
LIGO-confirmed the gravitational wave events detected by the LIGO-Virgo collaboration, and in the process of doing so,
claim to have discovered several others events [@IAS0; @IAS1; @IAS2]. Some of these events have total masses $>85
M_{\odot}$, which is larger than the average total mass of the LIGO detections. Some of these IAS and LIGO events are
displayed in Table~\@ref(tab:O2significancesWObcr) with their p-astro reported by various LIGO and IAS search pipelines.

```{r O2significancesWObcr, tidy=FALSE, echo=FALSE}
data <- read.csv('data/significances_without_bcr.csv',  stringsAsFactors = FALSE)
knitr::kable(
  data, booktabs = TRUE, escape =F, 
  caption = "p-astro from seveal detection pipelines for a subset of the O2 foreground triggers.",
  caption.short = "p-astro for various O2 foreground triggers" 
)
```

From this Table~\@ref(tab:O2significancesWObcr), it is evident there is some uncertainty if these events can be considered real
gravitational-wave events -- are these events significantly different from the background or not? The various pipelines
have different answers. Our Bayesian search pipeline, as described in the next section, can help us answer this question 
and come to a conclusion about the significance of the events.

### Using the BCR to calculate significance
While the BCR might not increase the significance of some of the gravitational wave candidates, it will provide a more
accurate measure of their significance. The BCR significance will be more accurate as it is a measure of the probability
that signals in two or more detectors are _coherent_ (signals share exactly the same source properties in the different
detectors) vs the detectors having _incoherent_ glitches/noise. This Bayesian significance might alleviate the discord
between the significances of specific foreground triggers reported by various search pipelines.

In summary, we will use the BCR to analyse high-mass foreground triggers to help discern signals from glitches and
provide a better measure of significance for the foreground triggers. For our work, we will initially focus on
conducting this BCR search on O2 data, and subsequently with a search in O3 data.



# Methodology {#methodology}
This PhD project intends to provide new methods and insights to studying gravitational waves form high-mass systems.
Thus far, the project has focused on building up the infrastructure to compare the Bayesian Coherence Ratio of LIGO's
time-shifted _background_ triggers, _foreground_ triggers and _software injections_. This chapter describes how we have
adapted the methods from @bcr_paper in our implementation.


## Selecting O3 Triggers for BCR analysis

LIGO's third observing run, O3, began on April, 2019, and is going until May 2020 (with a commissioning break during
October). In the 35 observing weeks since O3 started, based off the $\sim50$ public alerts for gravitational wave
candidate triggers, LIGO might have tripled the number of detections during all of the first and second observing runs,
O1 and O2 [@gracedb]. These gravitational wave events are considered candidates, as LIGO is still in the process of vetting the
candidates (making sure that they are from astrophysical sources rather than terrestrial, spurious noise sources). On
top of these candidate triggers, there are numerous sub-threshold triggers (triggers that do not have high
significance).

Ideally, we would like to compute the BCRs for all triggers (foreground and background) produced by LIGO to help improve
the significance of astrophysically real triggers and decrease the significance of glitches. However, computational
limitations prevent analysis of all triggers. Instead, we limit the data to several 8-day chunks from O2, selected based
on the number of foreground triggers identified by [`PyCBC`] in the various chunks, as seen in
Figure~\@ref(fig:o2TrigCount). Currently, we are using O2 data as it is public, whereas O3 data is not. By analyzing O2
triggers we can publish our results without any restrictions imposed by LIGO.

(ref:o2TrigCount) Counts of triggers from `PyCBC`'s search through O2 Data (November 15, 2016 - August 26, 2017). Each chunk is roughly 8 days long. The background triggers are collected from timesliding each chunk's data to generate a dataset of roughly 3 million years.

```{r o2TrigCount, echo=FALSE, fig.pos= "!h", fig.asp=.7, fig.width=6, fig.align='center', out.width='75%', fig.cap='(ref:o2TrigCount)', fig.scap="Count of all triggers in PyCBC O2 search"}
knitr::include_graphics("images/O2_unfiltered_trigger_counts.png")
```

Additionally, to confine our search to high-mass sources (systems with total masses roughly between 50-500
$M_{\odot}$), we restrict the triggers corresponding to a gravitational wave template with durations less than 454 ms.
These short duration templates correspond to compact binary coalesences with the parameters presented in
Table~\@ref(tab:parameters) and the search space for these triggers can be visualised in Figure~\@ref(fig:templateBank).

```{r parameters, tidy=FALSE, echo=FALSE, fig.pos= "!h"}
data <- read.csv('data/454ms_parameters.csv',  stringsAsFactors = FALSE)
knitr::kable(
  head(data), booktabs = TRUE, escape =F,
  col.names = c(" ","Minimum","Maximmum"),
  caption = "Template Banks's parameters for templates with duration $<454$ ms.",
  caption.short = "BBH parameters corresponding to durations $<454$ ms"
)
```

(ref:templateBank) The template bank used by [`PyCBC`] to search in O2's chunk 14. Our search is constrained to the high-mass parameter space enclosed by the dashed line (triggers with diurations $<45s$.  

```{r templateBank, echo=FALSE, fig.pos= "!h", fig.asp=.7, fig.width=6, fig.align='center', out.width='75%', fig.cap='(ref:templateBank)', fig.scap="High-mass BCR search space."}
knitr::include_graphics("images/template_bank_masses.png")
```

By restricting the [`PyCBC`] triggers only based on the template duration, we acquire common (high FAR) and rare (low
FAR) triggers. It would be straightforward in principle to broaden our template-duration constraint to encompass all
triggers produced by all the various search pipelines. However, we refrain from doing so to keep our computational costs
manageable. A plot of the counts of the triggers in O2 after downsampling the triggers based on the template duration
can be viewed in Figure~\@ref(fig:o2FilteredTrigCount).


(ref:o2FilteredTrigCount) Counts of triggers from `PyCBC`'s search through O2 Data, filtered to include only the triggers with a duration less than 454 ms

```{r o2FilteredTrigCount, echo=FALSE, fig.pos= "!h", fig.asp=.7, fig.width=6, fig.align='center', out.width='75%', fig.cap='(ref:o2FilteredTrigCount)', fig.scap="Count of triggers in PyCBC O2 search shorter than 454 ms"}
knitr::include_graphics("images/O2_filtered_trigger_counts.png")
```




## BCR Analysis Procedure 

### Computational setup
To compute the evidence making up the BCR (described in Section~\@ref(bcrCalculation)), we use [`bilby_pipe`] to acquire
the LIGO data in 4s-long data segments containing each trigger (analysis segments) and run the nested-sampling algorithm
implemented in the `dynesty` library. We utilise [`bilby_pipe`] to generate PSDs via Welch's method for each data
segment by taking 128 seconds of the data before the analysis segment. If the data quality is not marked as 'Science
Mode' for either the PSD or analysis data, then the analysis of that trigger is skipped. Gravitational-wave templates
are produced using `IMRPhenomPv2` ^[`IMRPhenomPv2` is a standard phenomenological waveform model constructed in the
frequency domain that models the inspiral, merger, and ringdown (IMR) of a compact binary coalescence with aligned spins
[@khan2016frequency].]. Although gravitational wave templates that are created from numerical relativity simulations
incorporate more physics than `IMRPhenomPv2`, we still use this waveform as it is faster to evaluate than numerical
relativity models. 

### Bayesian Parameter Estimation

We restrict the priors for Bayesian parameter estimation on the masses such that we only consider signals that are less
than 454ms in duration, as presented in Table~\@ref(tab:parameters). The spins are set to be aligned with a range from
[0,1]. The prior on luminosity distance assigns probability uniformly in comoving volume, with an upper cutoff of 5 Gpc.
The [`PyCBC`] search that originally produced our set of triggers considered a more comprehensive range of masses and
spins than we do in the BCR computation for this project. To accommodate this, we prescreened the triggers to only allow
triggers with masses within our priors.  The run time on a single trigger using the [`dynesty`] implementation of nested
sampling, with $2,000$ live points and $100$ walkers is usually between $0.5-12$ hours (where the speed often depends on
the SNR of the data segment).


### BCR calculation {#bcrCalculation}
The Bayesian coherence ratio (BCR), as seen in @bcr_paper, is the odds between the hypothesis that the data comprise a
coherent, compact binary coalescence signal in Gaussian noise ($\mathcal{H}_S$), and the hypothesis that they instead
comprise incoherent instrumental features ($\mathcal{H}_I$)—meaning each detector has either a glitch in Gaussian noise
($\mathcal{H}_N$) or pure Gaussian noise ($\mathcal{H}_G$). For a network of $D$ detectors,

\begin{equation}
\mathrm{BCR} = \frac{\alpha Z^{S}}{\prod_{i=1}^{D}\left[\beta Z_{i}^{G}+(1-\beta) Z_{i}^{N}\right]}\ , (\#eq:bcr)
\end{equation}
where $Z^S$	is the evidence for $\mathcal{H}_S$, and $Z^{G}_i$ and
$Z^N_i$ are the evidences for $\mathcal{H}_{Gi}$ and $\mathcal{H}_{Ni}$,
in the $i^{\text{th}}$ detectors. The arbitrary weights $\alpha$ and $\beta$
parametrize our prior belief in each model:

*  $\alpha = P(\mathcal{H}_S)/P(\mathcal{H}_I)$
*  $\beta = P(\mathcal{H}_{Gi}|\mathcal{H}_I)=1-P(\mathcal{H}_{N}|\mathcal{H}_I)$

for all $i$ detectors. As most of the data should be noise, we expect both $\alpha$ and $\beta$, to be less than 1.
These parameters will be tuned to minimize the overlap between the signal and noise trigger populations.

### Trigger sets for BCR analysis

We calculate the BCRs using the procedure described above for three separate trigger sets: background triggers,
foreground triggers, and software injections. The software injections created using masses, spins and distances that
span our priors. The software injections are inserted in the O2 data from where the other triggers are retrieved.

We expect the background trigger BCRs to favour the incoherent glitch hypothesis, and the software injection BCRs to
favour the coherent, compact binary coalescence hypothesis. The foreground trigger BCRs should fall along with the
background if the foreground trigger is a glitch, and with the software injections if the foreground trigger is a
gravitational wave candidate. For example,  consider Figure~\@ref(fig:bcrIfar) that contains a plot of the BCRs for
several O1 background triggers, foreground triggers, and software injections plotted against iFAR (inverse
false-alarm-rate). In the figure, some foreground triggers fall along the background distribution, these are discarded
as glitches. The foreground triggers that fall along the bulk of software injections are considered gravitational wave
candidates.

(ref:bcrIfar) BCR versus iFAR distributions from simulated signals (blue), hardware injections (green), GW events (stars) and background triggers (black), taken from @bcr_paper.

```{r bcrIfar, echo=FALSE, fig.pos= "!h", fig.asp=.7, fig.width=6, fig.align='center', out.width='75%', fig.cap='(ref:bcrIfar)', fig.scap="BCR vs iFAR for O1"}
knitr::include_graphics("images/bcr_ifar.png")
```



### Making a significance statement

After obtaining the BCRs for the foreground triggers, background triggers, and software injections, we can make a
statement about the significance of foreground events. For example, consider Figure~\@ref(fig:bcrCdf) which displays ln
BCR distributions obtained from our BCR pipeline for aLIGO O2 PyCBC chunk three background triggers and software injections.
The figure also displays the ln BCR values obtained for four foreground triggers. The two foreground triggers with low
ln BCRs are confirmed glitches. The rightmost foreground trigger is GW170104, a gravitational wave event confirmed by
LIGO, and shows much stronger evidence for it being a coherent, compact binary coalescence signal, rather than it being
an incoherent glitch. The remaining foreground trigger corresponds to a gravitational wave candidate identified by the
IAS group.

(ref:bcrCdf) Histograms represent the survival function (1-CDF) from our selection of 2407 aLIGO O2 PyCBC chunk 3 background triggers (gray) and 648 simulated signals (blue). Vertical lines mark the ln BCRs of two glitches (orange and yellow), IAS's GW170121 (pink), and GWTC-1's GW170104 (dark blue). 

```{r bcrCdf, echo=FALSE, fig.pos= "!h", fig.asp=.7, fig.width=6, fig.align='center', out.width='75%', fig.cap='(ref:bcrCdf)', fig.scap="BCR distribution example"}
knitr::include_graphics("images/bcr_cdf.png")
```

Irrespective of the BCR's Bayesian interpretation, we may treat the BCR as a traditional detection statistic to obtain a
frequentist estimate of the significance of any given foreground event based on the measured background with a p-value.
The p-value of foreground events can be obtained from the background's survival function. For example, in 
Figure~\@ref(fig:bcrCdf), the p-value for GW170104 is 0.99, denoting that it is significantly different from the bulk
of the background distribution and hence likely to have astrophysical origins. On the other hand, the p-value for the
IAS candidate is 0.65, demonstrating that this candidate might be a glitch.



To calculate the p-value, we need first to calculate fits to the BCR probability
distributions and then tune the BCR's parameters. The following subsections describe these calculations.


#### Fitting BCR data

Although the BCR should favour the coherent signal hypothesis for software injections,
software injections with a very low SNR might not look coherent amongst an ensemble of
detectors. Hence, software injections are likely to have two probability distributions
of BCRs, one probability distribution of low BCRs corresponding to the software
injections that have very low SNRs and another probability distribution of software
injections with high BCRs. Due to this, we use Gaussian mixture models with two Gaussian
components to fit our BCR  distributions.

#### Tuning of BCR using KL Distance

The BCR has two tunable parameters as described in Section~\@ref(bcrCalculation),
$\alpha$ and $\beta$. Altering $\alpha$ translates the BCR probability distributions for
the triggers while adjusting $\beta$ spreads the probability distributions out. Although
Bayesian hyper-parameter estimation might be able to determine the optimal values for
these parameters, an easier approach is to manually adjust the parameters for each data
chunk's BCR distribution. In this study, we manually adjust $\alpha$ and $\beta$ to
maximally separate the BCR distributions for the software injections and background
triggers.

To calculate the separation between the background and software injection BCR
distributions, we use the Kullback–Leibler divergence (also called relative entropy, or
KL divergence) to quantify how different how the background and software injection  BCR
probability distributions are. If the distributions are identical, the KL divergence is
equal to 0, and the KL divergence increases as the asymmetry between the distributions
increases. Figure~\@ref(fig:klDivGrid) displays a contour plot of O2 Chunk 14's KL
divergences between the background and software injection triggers for values of 
$\alpha$  and $\beta$ ranging from $1^{-10}$ to $1$. We select these ranges for $\alpha$
and $\beta$ as values outside this range will not make physical sense. Note that for a
fixed value of $\beta$, the KL divergence stays the same as we alter $\alpha$. This is
because $\alpha$ translates the BCR values, but does not alter the distributions
themselves.

(ref:klDivGrid) KL Divergences for O2 Chunk 14's software injection and background trigger BCR probability distributions. The red line indicates the $\alpha$ and $\beta$ parameters with the maximum KL divergence in this range of $\alpha$ and $\beta$.

```{r klDivGrid, echo=FALSE, fig.pos= "!h", fig.asp=.7, fig.width=6, fig.align='center', out.width='75%', fig.cap='(ref:klDivGrid)', fig.scap="KL Divergence grid for both BCR tunable parameters"}
knitr::include_graphics("images/kl_divergence_grid.png")
```

Figure~\@ref(fig:klDivLine) displays a plot of O2 Chunk 14's KL divergences only as
$\beta$ is adjusted for a fixed $alpha=1^{-6}$. The $\alpha$ and $\beta$ are then selected
based on the combination that results with the maximum KL divergence.

(ref:klDivLine) KL Divergences for O2 Chunk 14's software injection and background trigger BCR probability distributions with $alpha=1^{-6}$.

```{r klDivLine, echo=FALSE, fig.pos= "!h", fig.asp=.7, fig.width=6, fig.align='center', out.width='75%', fig.cap='(ref:klDivLine)', fig.scap="KL Divergences for one BCR tunable parameter"}
knitr::include_graphics("images/kl_distance_14.png")
```

In Chapter~\@ref(progress) we present some of the preliminary BCR search results we 
have obtained from the analysis of various O2 data chunks.  





[`dynesty`]: https://arxiv.org/abs/1904.02180
[`bilby_pipe`]: https://lscsoft.docs.ligo.org/bilby_pipe/


## Definition of an IMBH

In 1932, Landu and Chandrasekhar theorized the end point of massive
stars that collapse into singularities as Black holes (BHs). In the
1970s, X-ray and optical observations from the Cyg-X1 binary star system
lead to the discovery of the first stellar mass black hole. Shortly
afterwards in 1974, kinematic measurements of stars near the center of
our galaxy demonstrated the presence of an SMBH of mass
$10^6\ \text{M}_{\odot}$ -- Sag A* [@sagA].

These two detections firmly established the existence of stellar mass
black holes $10^{2} \ \text{M}_{\odot} > \text{M}_\text{BH}$ and
supermassive black holes (SMBH), $10^{5} \ \text{M}_{\odot} <
\text{M}_\text{BH}$ in our universe. Since then, more evidence for
various black holes has been acquired. However, none of the bono fide
black holes have masses between $10^{2} - 10^{5}\ \text{M}_{\odot}$. Hence,
we have yet to confirm if Intermediate mass black holes (IMBHs) in the
$10^{2} - 10^{5} \ \text{M}_{\odot}$ range exist.
Figure~\@ref(fig:mass-chart) displays the three categories of BHs
we have discussed.

(ref:mass-chart) A mass chart for dead stars and black holes, taken from @BHmassChartNasa.

```{r mass-chart, echo=FALSE, fig.pos= "!h", fig.asp=.7, fig.width=6, fig.align='center', out.width='90%', fig.cap='(ref:mass-chart)', fig.scap="Compact object mass ranges"}
knitr::include_graphics("images/mass_chart.jpg")
```



## Formation Scenarios

To help reflect on the various formation scenarios for IMBHs,  
as suggested by @miller2003formation,
one can recall the following Shakespearean quotation:

> "... some are born great, some achieve greatness, and some have greatness thrust upon 'em. "
>
> --- Shakespeare:  Twelfth Night, Act II Scene V

This quote inadvertently summarises the popular IMBH formation
scenarios: some IMBHs might form directly from the collapse of massive
stars or pristine-gas galaxies [@lodato2006supermassive], some IMBH
might start as stellar mass black holes that accrete enough mass to fall
into the intermediate regime [@coleman2002production], and finally some
IMBH might be created via repeated mergers of stellar mass black holes
[@2018IMBHreview]. These scenarios along with some others are discussed
in Section~\@ref(scenarios). Section~\@ref(locations) provides
hypotheses on where these IMBHs might be located in our universe after
formation.

### Different Scenarios {#scenarios}

There are various postulated scenarios of IMBH formation. The scenarios
discussed in this section are displayed in the schematic diagram
presented in Figure~\@ref(fig:imbh-sources). These formation channels
can be divided into two main groups based on the seeds that lead to the
IMBH to form: light and heavy seeds. for a more comprehensive list and
details of IMBH formation scenarios, refer to @van2004intermediate.

(ref:imbh-sources) Schematic diagram of various formation scenarios of IMBHs and SMBHs (figure adapted from @mezcua2017observational).

```{r imbh-sources, echo=FALSE, fig.pos= "!h", fig.asp=.7, fig.width=6, fig.align='center', out.width='90%', fig.cap='(ref:imbh-sources)', fig.scap="IMBH formation channels"}
knitr::include_graphics("images/imbh_sources.png")
```

<!--####  Context and formation of IMBHs-->



<!--2.1. Suggested formation mechanisms for IMBHs-->
<!--The reason for setting the lower limit on IMBHs at ∼100 M is that this-->
<!--appears to be in excess of the maximum black hole mass that can form-->
<!--from a solitary star in the current universe (see [17] for a discussion-->
<!--in the context of IMBHs). IMBH existence therefore requires new-->
<!--formation scenarios. The three basic ideas are-->

<!--• The first generation of-->
<!--stars had negligible metallicity. This reduces radiative opacity and-->
<!--thus the strength of radiation driven winds, hence stars can start more-->
<!--massive and retain more of their mass than current stars [17]. • In-->
<!--young massive stellar clusters that have relaxation times for the most-->
<!--massive stars that are less than the lifetimes of those stars (about 2.5-->
<!--million years), mass segregation leads to collisions and mergers of-->
<!--massive stars. If the combined objects do not have catastrophic wind-->
<!--mass loss (see [4]), they can, in principle, accumulate up to thousands-->
<!--of solar masses and might then become black holes with similar masses-->
<!--[6, 12, 31, 33].-->

<!--• If a seed black hole with a mass of more than ∼200 M-->
<!--is formed by the previous process in a dense stellar cluster, subsequent-->
<!--binary–single and binary–binary interactions can allow it to merge and-->
<!--accumulate mass without being vulnerable to ejection by either-->
<!--three-body kicks or recoil from asymmetric gravitational wave emission-->
<!--[10, 11, 25, 26, 28, 29]. Of special interest to LISA observations is-->
<!--that some simulations suggest that if the initial binary fraction in a-->
<!--stellar cluster exceeds ∼10% then more than one IMBH can form in that-->
<!--cluster ([12]; but see [4]). We will explore the consequences of this in-->
<!--section 3.1. In addition, since massive stellar clusters are often found-->
<!--near the nuclei of galaxies that are interacting actively, the clusters-->
<!--themselves can sink to the centers of galaxies where their IMBHs spiral-->
<!--into the galaxy’s supermassive black hole. We discuss this in section-->
<!--3.2.-->



#### Light Seed

The light-seed formation channels discussed in this section are
displayed in the low-mass region, $10^{1-2} \ \text{M}_{\odot}$, of
Figure~\@ref(fig:imbh-sources). These channels are:


1. *Accretion onto a stellar mass black hole*:
    BHs can accrete mass from its surrounding at the Eddington rate.
    This process can increase the BHs mass. Hence, gradual accretion
    onto a stellar mass BH could lead to the formation of an IMBH.
    However, the rate of accretion is small, leading to growth timescales
    longer than the time needed for a BH to cross a molecular
    cloud, or even the lifetime of such a cloud [@coleman2002production].


2. *Population III stars*:
    Population III stars are a hypothetical, extremely hot and massive stars
    (with masses $10^{2-3}\ \text{M}_{\odot}$) that form from metal-free
    primordial gases [@bromm2013formation]. @heger2003massive point out that
    there is a `mass-gap` between $140-260 \ \text{M}_{\odot}$ a region
    where if a Population three star collapses, it has no remnant. This mass
    gap is depicted in Figure~\@ref(fig:massgap). Only if the mass of the
    star is above $260 \ \text{M}_{\odot}$ can it form a BH of at
    least half of the original star's mass (and hence form an IMBH).

(ref:massgap) Remnant of massive stars as a function of initial metallicity versus initial mass, taken from @heger2003massive.

```{r massgap, echo=FALSE, fig.pos= "!h", fig.asp=.7, fig.width=6, fig.align='center', out.width='90%', fig.cap='(ref:massgap)', fig.scap="Plot of metallicity versus mass depecting the mass gap" }
knitr::include_graphics("images/hegar_mass_gap.jpg")
```

3. *Runaway-collisions in Low-Metallicity Clusters*:
   @vanbeveren2009stellar simulated the dynamics of young dense stellar
   systems including the effects of stellar wind mass loss. With the
   inclusion of stellar-wind mass loss they were able to model
   successive collisions of the most massive stars in the stellar
   systems. These collisions, known as runaway-collisions, formed very
   massive stars (with mass $> 120 \ \text{M}_{\odot}$) in their
   simulations. These $> 120 \ \text{M}_{\odot}$ mass stars left behind
   BHs with mass $\approx 70\ \text{M}_{\odot}$, and in low metallicities,
   BHs with mass $\approx 100\ \text{M}_{\odot}$ (at the edge of the IMBH
   regime). These runaway-collisions IMBHs could hypothetically form in
   the nuclear cluster of $2^\text{nd}$ generation stars
   [@mezcua2017observational].


4. *Mergers of stellar mass black holes*:
    In old stellar clusters ($\sim10^8$ yrs old), most of the massive
    objects are stellar remnants (the end stages of evolution for
    stars). For many years it had been hypothesised that BHs could form
    binaries in such clusters and eventually merge. In 2015, LIGO proved
    the existence of binary BHs (BBHs) by detecting gravitational waves
    generated by the merger of a BBH [@abbott2016observation]. The
    remnant mass of this merger was $\sim63\ \text{M}_{\odot}$. Since
    then LIGO has detected several more GWs from BBHs (see
    Figure~\@ref(fig:o1o2) for a visual representation of the various GW
    detections from the first and second LIGO observing runs). Thus, it
    might be possible to detect gravitational waves from the merger of a
    BBH that results in the formation of an IMBH, or even an IMBH-BH
    merger [@salemi2019search]. The expected rates of such mergers are
    discussed in Section~\@ref(motivations).

    The difficulties with forming IMBHs through BH mergers are that the
    escape velocities of globular clusters are very low at $\sim50$km/s
    [@morscher2013retention]. Thus, if another compact object passes by
    the binary, the binary might get ejected from the globular cluster.
    Although the ejected binary will still be able to merge, the
    post-merger remnant will be unable to undergo a $2^{\text{nd}}$
    generation merger (due to the lack of other BHs in its vicinity),
    leading to a smaller chance of accumulating mass via mergers.  In a
    similar vein, if the BBH does manage to merge inside the globular
    cluster, gravitational waves from the coalescence of the BBH  will
    carry away linear momentum, causing the renmant post-merger BH to
    get a get a recoil. This "kick" effect has could lead to the
    ejection of the post-merger BH from the globular cluster
    [@favata2004black].

    It is important to note that if one of the BH's has an initial mass
    $>50 \ \text{M}_{\odot}$ while the other has a comparatively smaller
    mass, it is possible for the larger BH to undergo successive mergers
    and stay inside the cluster due to inertia [@coleman2002production].

(ref:o1o2) Diagram of the various detections in O1 and O2, created by Boyand Liu.

```{r o1o2, echo=FALSE, fig.pos= "!h", fig.asp=.7, fig.width=6, fig.align='center', out.width='75%', fig.cap='(ref:o1o2)', fig.scap="Detections in GWTC-1"  }
knitr::include_graphics("images/o1o2_detections.png")
```




#### Heavy Seed

The heavy-seed formation channels discussed in this section are
displayed in the high-mass region, $10^{5-8} \ \text{M}_{\odot}$, of
Figure~\@ref(fig:imbh-sources). The two heavy-seed channels in the
Figure both involve the first metal-free protogalaxies.

In the first channel, if fragmentation is prevented, rapidly inflowing
dense gas of the protogalaxy can form a supermassive star of  $\sim
10^{5} \ \text{M}_{\odot}$ [@regan2014direct]. The collapse of such a
large star can form a black hole of comparable mass [@regan2014direct].

The second channel considers the mergers of metal-free protogalaxies.
Mergers of metal-free protgalaxies can form massive compact nuclear
gas disks, with a supermassive star at its center. This star can
collapse to form an IMBH or even an SMBH of  $\sim10^{8} \
\text{M}_{\odot}$ [@mezcua2017observational].



### SMBH formation via IMBHs

We have evidence that SMBHs formed at $z\sim7$ or 0.8 Gyr after the
formation of the Universe [@mezcua2017observational]. However, we are
unsure of how these SMBHs form. A popular notion was that stellar mass
BHs that formed soon after the Big Bang accreted enough mass to form
into these SMBHs. However, if stellar mass BHs obey the Eddington
Limit, 0.8 Gyr is too short a time frame for stellar mass BHs to
accrete enough mass to turn into SMBHs. This is where IMBHs come in --
instead of stellar mass BH, there might have been IMBH in the early
universe as the seeds of SMBHs. Hence several formation scenarios of
SMBHs propose that they pass through a life stage in which they are
IMBHs [@mezcua2017observational]. Additionally, the IMBHs which do not
turn into SMBHs should still be present as leftover IMBHs.


Based on these concepts, we can try to speculate where we might find
IMBHs.


### Current Locations of IMBHs {#locations}

According to @mezcua2017observational, leftover IMBHs that did not turn
into SMBHs should be found in dwarf galaxies, globular clusters, young
star clusters, and finally, some might be found orphaned from globular
clusters.

One possible formation channel for IMBHs we discussed were from multiple
BH mergers. If the IMBH that forms from the merger manages to stay
inside the globular cluster, IMBH might form an active galactic nuclei
(AGN). If this IMBH's host galaxy is close to other galaxies hosting
IMBHs, then the galaxies might merge, leading to an IMBH merger.
Subsequent IMBH mergers like this would turn the IMBH into an SMBH.
However, if the galaxy is an isolated dwarf galaxy, then the IMBH might
be able to remain as an IMBH. Hence, IMBHs might be found at the center
of globular clusters, or in isolated dwarf galaxies.

Another formation channel involved runaway collisions of stars with low
mentalities. If such runaway collisions can occur in regions with no
stellar winds, then IMBHs might form in young star clusters where these
collisions take place.

Some examples of candidate galaxies with IMBHs can be seen in
Figure~\@ref(fig:example-locations).

(ref:example-locations) The globular cluster 47~Tucanae, the dwarf galaxy NGC~4395 and the spiral galaxy ESO~243-49 are all host-galaxies for candidate IMBHs.

```{r example-locations, echo=FALSE, out.width='30%', fig.show='hold', fig.align='center', fig.cap='(ref:example-locations)', fig.scap="Candidate IMBH host-galaxies"}
knitr::include_graphics(c("images/47tuc.png", "images/ngc4395.png", "images/eso243.png"))
```



## Motivations for finding  {#motivations}

Even though the existence of IMBHs is debatable, these objects if
discovered, can play a pivotal role in understanding various
astrophysical processes. If an IMBH were to be observed our
understanding of the formation and evolution of SMBHs, as well as galaxy
evolution modeling and theories of strong-field regime of gravity could
dramatically change [@mezcua2017observational]. We might also understand
more about tidal disruption of stars [@IMBHcauseTidalDisruption], and
the quenching of star formation in early galaxies [@kim2011galaxy].
These, along with some other motivations for detecting IMBHs are
discussed in this section.


1) _SMBH and galaxy evolution_

    Detecting IMBHs will be pivotal in understanding SMBH and galaxy
    evolution. As mentioned earlier, most formation scenarios of SMBHs
    propose that they pass through a life stage in which they are IMBHs
    [@mezcua2017observational].

    Understanding the formation of IMBHs and SMBHs will in turn yield
    insights to the early structure of the Universe. For example, the
    number of present-day IMBHs can elucidate how seed BHs formed in the
    early Universe: either from the death of Pop III stars, the direct
    collapse of pristine gas, or by stellar mergers in dense stellar
    clusters (each have different populations for IMBHs).


2. _Test of GR_

    The detection of the gravitational capture of a stellar-mass compact
    object by an IMBH will allow us to test gravity in the strong regime
    [@GWimbhProspects].



3) _Source of gravitational waves_

    @abadie2010ratePredictions noted that the portions of GWs of certain
    IMBH binaries (IMBHBs) and IMBH-BH binaries should be detectable in
    the advanced LIGO and VIRGO interferometers.

    Simulations by @fregeau2006imbhbRatePrediction have demonstrated
    that IMBHBs might form during runaway collisions scenarios in young
    star clusters. According to the simulations, the recently formed
    IMBHB quickly tighten into a common binary shortly after their birth
    and quickly inspiral and merge. The rate of such IMBHB mergers
    detectable by aLIGO and VIRGO is theorized from this simulations to
    be $\sim10\ \text{yr}^{-1}$. @abadie2010ratePredictions points
    out another formation channel for IMBHBs: from after the
    collision of two globular clusters (each with an IMBH). However,
    no rates for such IMBHB mergers exists.

    In a similar vein, simulations of interactions inside globular
    clusters demonstrate that IMBH-BH pairs with low mass-ratios may
    form~[@mandel2008rates]. Such binary systems might undergo
    tightening through various three body interactions, and eventually
    merge. Due to the extreme mass ratio of this scenario, the scenerio
    is named an intermediate mass-ratio coalescence, IMRAC. According to
    @mandel2008rates such events can occur anywhere from 0-10 times a
    year.

The rates for such events are tabulated in Table~\@ref(tab:ligo-rates).


(ref:imbhRatePrediction) @fregeau2006imbhbRatePrediction

(ref:mandel2008rates) @mandel2008rates

(ref:rodriguez2015bbhRatePredictions) @rodriguez2015bbhRatePredictions

```{r ligo-rates, tidy=FALSE, echo=FALSE}

data <- read.csv('data/imbh_rates.csv',  stringsAsFactors = FALSE)

knitr::kable(
  head(data), booktabs = TRUE, escape =F,
  col.names = c("Black Holes","Merger Type","Event Rate $r$ [1/yr]","Reference"),
  caption = 'A table describing the different detection rate estimates in the aLIGO band.',
  caption.short = "High-mass detection rates in aLIGO" 
)
```


<!--A match-filtered-->
<!--observation of the inspiral would yield the redshifted masses-->
<!--of the black holes, directly confirming the existence of IMBHs.-->
<!--It would also yield the luminosity distance to the source; with-->
<!--enough observations, constraints could be placed on the cosmic-->
<!--history of star formation in dense, massive clusters. Detection-->
<!--of the ringdown signal from the merger product will also yield-->
<!--its spin, which may provide insight into its formation history-->



## Observational Methods

To make a definitive detection of an IMBH, a clear measurement of the
IMBH's mass is required [@koliopanos2018intermediate].  There
are two broad categories for observational methods employed to detect
IMBHs, direct and indirect observational methods.

### Direct observational method

Some of the direct methods used to measure SMBH masses, such as dynamic
kinematic measurements and reverberation mapping, have been employed
to measure the masses of various IMBH candidates. However, none of these
studies have proved conclusive as the sphere of influence of IMBHs are
too small to be resolved.

Dynamic kinematic measurements of individual sources orbiting a BH, such
as kinematic measurements of orbits around Sgr A* (see
Figure~\@ref(fig:sag-a)), is the most reliable method of measuring BH
mass. @47TucHostIMBH studied the motion of 19 millisecond pulsars in the
globular cluster 47-Tucanae and described their motion using an IMBH
with mass ${\sim2300\ \text{M}_{\odot}}$. However, others were able to
describe the motions without an IMBH [@freire2017long].

(ref:sag-a) Inferred orbits of 6 stars around Sgr A*, giving us mass of the SMBH [@sagA].

```{r sag-a, echo=FALSE, fig.pos= "!h", fig.asp=.7, fig.width=6, fig.align='center', out.width='75%', fig.cap='(ref:sag-a)', fig.scap="Inferred orbits around Sgr A*"}
knitr::include_graphics("images/sagitarius_a.png")
```

Reverberation mapping is a procedure where time-delays of radiation from
a BH and its orbiting gas can be used to infer mass of the BH. A
schematic diagram in Figure~\@ref(fig:revmap) illustrates this concept
this concept. @ncg4395 analysed the timing delays from an IMBH candidate
in Dwarf galaxy NGC-4395 and estimated that a IMBH with mass
${\sim3.6\times10^5\ \text{M}_{\odot}}$ was the source of the time-delay
radiation. However, because we cannot discount the possibility that the
radiation from the source could be anisotropic or non-linear (leading to
overestimating the true source mass), reverberation mapping by
itself is inadequate.


(ref:revmap) Radiation received from a direct path after scattering off moving clouds that surrounds the black hole allows inferring the velocity of clouds and hence mass of SMBH.

```{r revmap, echo=FALSE, fig.pos= "!h", fig.asp=.7, fig.width=6, fig.align='center', out.width='90%', fig.cap='(ref:revmap)', fig.scap="Reverberation mapping schematic"}
knitr::include_graphics("images/revmap.png")
```

Another drawback of direct methods are that they cannot probe beyond
only the most nearby galaxies. To be able to study even the smaller, less
luminous, and further away IMBHs, indirect methods of observation can be
used.

### Indirect Observational Methods

There are several indirect observational methods that have been employed
to search for IMBHs (see @mezcua2017observational for a detailed
study on the various methods and their results). Here, we discuss
looking for IMBHs by studying low-luminosity AGNs, ultraluminous x-ray
sources, and gravitational waves.

#### Studying x-ray emission
AGNs create unique optical spectra due to the ionisation of gas,
surrounding the BH, from x-rays released from accretion discs
@elvis2000structure. When the emission lines from this optical spetra are
studied alongside the broad-line emission lines from the central BH of
the AGN, a virial estimate of the BH mass can be calculated
[@IMBHinLLAGN]. Although several groups have claimed to classify IMBHs
by studying such AGNs, the results have large uncertainties and are
contradictory.

In a similar vein there have been numerous studies of x-ray emissions
(from off-nuclear extragalactic sources) exceeding the Eddington limit
for stellar mass BH [@kaaret2017ultraluminous]. These sources, known as
ultraluminous X-ray sources (ULXs) make a large IMBH candidate set
(under the assumption that the emission is isotropic). One such IMBH
candidate labeled HLX-1 located in the galaxy ESO~243-49 is considered
to have a mass of $\text{M}\sim5000\ \text{M}_{\odot}$ due to a luminosity of
$L_x\sim10^{42}\ \text{erg s}^{-1}$ [@hlx1].

#### Studying GWs
Another pathway to detecting an IMBH is with gravitational waves emitted
from compact binary black hole systems. As the frequency at which
gravitational waves are emitted is inversely proportional to the chirp
mass of the system, higher mass systems correspond to lower frequency
gravitational waves [@abbott2016properties]. Currently, there are three
ground based detectors (LIGO-Hanford, LIGO-Livingston and Virgo) that
are operational and have successfully made GW detections of binary BH
(BBH) and binary neutron star (BNS) mergers [@abbott2019gwtc]. As
depicted in Figure~\@ref(fig:sensitivity), these detectors are not
sensitive to supermassive ($\sim10^9\ M_{\odot}$) and massive
($\sim10^{4-7}\ M_{\odot}$ binary systems. However, as discussed
in Section~\@ref(motivations), these detectors might be able to detect
GWs from an IMBH system with a total mass less than $500\
\text{M}_{\odot}$ [@salemi2019search].


(ref:sensitivity) Approximate signal ranges and sensitivities for some potential GW sources and GW detectors. Figure was created using the tool provided by @moore2014gravitational.

```{r sensitivity, echo=FALSE, fig.pos= "!h", fig.asp=.7, fig.width=6, fig.align='center', out.width='90%', fig.cap='(ref:sensitivity)', fig.scap="Gravitational wave detector sensitivities."}
knitr::include_graphics("images/sensitivity.png")
```

In July 2019, the LIGO-Virgo scientific collaboration released the
outcomes of an IMBH search they conducted on their first and second
observing runs. Although their work did assist in constraining IMBHB
merger rates, they were unable to detect an IMBH binary merger
[@salemi2019search].

A potential reason they provided for why no IMBHBs were found was
because of detector noise. As IMBHs have high mass, the IMBH GW signals
have a low frequency and a duration lasting less than a second
[@salemi2019search]. This corresponds to a short-duration trigger in the
LIGO data. A trigger is a section of LIGO data identified by data
searching pipelines to potentially harbour some GW signal data (more on
triggers in Section~\@ref(searches) where the search pipelines are
discussed). In the ground based GW detectors, there are many spurious
instrumental noise events, that fool the pipelines into considering them
as triggers. These instrumental-triggers are called glitches. Luckily,
unlike triggers from astrophysical sources, glitches do not often occur
at the same time or with the same morphology across the various GW
detectors -- they are often incoherent. However, as glitches appear very
similar to expected short duration IMBH signals, it can be difficult
distinguishing the two. For example, consider Figure~\@ref(fig:compare)
that compares a plot of a glitch on the left and a plot of a high-mass
GW candidate on the right.


(ref:compare) Spectrogram of a glitch on the left, and a candidate GW trigger on the right.

```{r compare, echo=FALSE, fig.show='hold', fig.asp=.7, fig.width=6, fig.align='center', out.width='45%', fig.cap='(ref:compare)', fig.scap="Comparing glitches to high-mass candidates."  }
knitr::include_graphics(c("images/glitch.png", "images/L1_spectrogram_logy_0.png"))
```


@salemi2019search stated that increases of
the GW detector network sensitivity and search techniques in future
runs might result in a detection of GWs from an IMBH. The following
Section~\@ref(searches) discusses some of these searches, and introduces
a novel search for IMBHs that can better discern between IMBH signals
and glitches.



## Current searches for GWs {#searches}
As LIGO records data, generic unmodelled GW transients and modelled GW signals are
searched for in the data using low-latency search pipelines. These searches can identify
candidate GW events in near-real-time timescales, enabling the possibility of performing
follow-up electromagnetic spectrum observations [@abbott2018prospects]. Using such 
searches, the LIGO-Virgo collaboration was able to identify 11 gravitational wave 
sources in the first and second observing runs of Advanced LIGO [@abbott2019gwtc]. 
External searches conducted by groups such as the Institute for Advanced Study have added
10 other potential gravitational wave sources [@IAS0; @IAS1; @IAS2; @UdallIMBHsearch].
In this section we delve into some low-latency LIGO searches and
non-LIGO searches, and discuss the differences between them.

### LIGO's Unmodelled GW Searches {#unmodeledSearches}
The unmodelled searches are performed without prior information about
expected GW signal waveforms. These searches look for statistically
significant regions of the data with excess-power, or bursts, and are
effective at identifying GW transients that can last for
$10^{-3}-10^{1}$s [@abbott2016observing]. Some examples of such short
duration GWs are from the coalescence of compact binary-black holes,
from unknown sources, and from other astrophysical phenomena as
described in [@abbott2018prospects].

Two unmodelled-generic search piplines for GWs are the Coherent
WaveBurst  pipeline, [`cWB`], and omicron-LALInference-Bursts pipeline,
[`oLIB`]. Both pipelines first locate coincident events in data
from the online GW detectors (using different methods). Following
the tabulation of coincident events, the pipelines provide
a significance for the coincident events in  unique ways.

Once coincident events are obtained, the [`cWB`] pipeline  uses a
likelihood analysis to reconstruct the GW signals associated with the
unique events. Once the best waveform (similar in both
detectors) has been constructed, the reconstructed signal is subtracted  
from the data. The remaining energy after the subtraction is typically  
 low for coherent GW events, and high for incoherent and spurious  
 instrumental noise events (glitches).

 On the other hand, the [`oLIB`] pipeline runs Bayesian parameter
 estimations for all the coincident events by modelling GW signals and
 glitches using sine Gaussian wavelets. It then calculates two Bayes
 factors, one to compare the coherent signal versus Gaussian noise
 hypothesis, `BSN`, and another to compare the coherent signal versus
 incoherent glitch hypothesis, `BCI`.


@abbott2016observing provides a detailed summary of these two pipelines
in relation to GW150914's detection.

### LIGO's Modelled GW Searches {#modeledSearches}
The previous section discusses how generic searches can be used to
look for GWs from unknown sources. This method is used because  
we do not know what GWs we should expect from unknown sources.

In contrast, there have been numerous studies and predictions for what
GWs we expect from the coalescence of compact binary coalescence (CBCs)
including GWs from BNS, BBH and NSBH mergers [@abbott2016ligo]. Modelled
GW searches such as templated searches for transient GWs use prior
information about our expectations of GWs into searches. Two templated
searches for GWs from CBCs are the [`PyCBC`] [@biwer2019pycbc], and
[`GstLAL`] pipelines [@sachdev2019gstlal].

The [`PyCBC`] and [`GstLAL`] use match-filtering, a method of extracting
GW signals from noisy data, by cross-correlating the data from different
detectors and a predicted GW waveform signal --  a template
[@abbott2016ligo]. Both pipelines use the same waveform templates, but
use different techniques to apply the match-filtering. They also
implement unique methods of estimating the background and weeding out
glitches from astrophysically significant coincident triggers.




## Ranking statistics of Current Searches

All the current searches discussed in Section~\@ref(searches), although
implemented in different ways, ultimately use a `ranking statistic` to
make statements about the significance of certain triggers over others.
In principle, to make a rigorous statement about the statistical
significance of a pair of coincident triggers, it is necessary to know
the probability that a given trigger was produced by instrumental noise,
rather than an actual GW. This likelihood may be estimated empirically
from the value of the ranking statistic for a large representative set
of triggers known with certainty to be spurious. Such a set of
signal-free triggers is denoted _background_, in contrast to the
_foreground_ of candidates that may contain a signal.


Because detectors cannot be physically shielded from gravitational
waves, ad hoc data analysis techniques must be used to estimate the
background. One such strategy is to construct time slides by applying
relative time offsets (longer than the light-travel time between sites)
between the data of different detectors. An schematic illustration of
such a time slide is demonstrated in \@ref(fig:timeslide). Detection
significance can then be inferred, in a frequentist way, by comparing
the value of the ranking statistic for a time-coincident foreground
trigger to that of time-slid background triggers. The rate at which
background triggers are produced with a given value of the ranking
statistic is usually referred to as the _false-alarm rate_ (FAR).


(ref:timeslide) Timeslide schematic diagram: coincident data on the left time-slid data on the right.

```{r timeslide, echo=FALSE, fig.show='hold', fig.asp=.7, fig.width=6, fig.align='center', out.width='45%', fig.cap='(ref:timeslide)', fig.scap="Timeslide schematic"  }
knitr::include_graphics(c("images/t0.png", "images/t1.png"))
```

Efficient signal detection requires a ranking statistic that extracts
the most information from the data to discriminate between
noise and weak astrophysical signals. However, existing CBC searches are
not optimal in this sense: they do not incorporate knowledge of all
features that may distinguish GWs from noise. Moving towards an optimal
statistic is a great challenge, but one large step is to demand that
foreground triggers in two or more detectors should be better described
as coherent gravitational-wave signals, rather than incoherent glitches.
Importantly, it is not enough to provide some measure of coherence: one
must also prove that an incoherent model is not more successful at
describing the data. To achieve this, the Bayesian coherence ratio,
`BCR`, proposed by @bcr_paper can be utilised.

## Bayes' Theorem

Before introducing the BCR, it is necessary to understand Bayes' theorem. Bayes' theorem allows us to use some knowledge
that we already have about some parameters of a mathematical model, the _prior_ $\pi(\theta)$, to help us calculate the
probability of getting the parameters given some data, the _posterior probability_ $p(\theta|d)$.  Mathematically,
Bayes' theorem can be written out as

\begin{equation}
{p(\theta|d)} = \frac{\mathcal{L}(d|\theta)\pi(\theta)}{\mathcal{Z}}\ , (\#eq:bayeTheorem)
\end{equation}

where $\mathcal{L}(d|\theta)$, or the likelihood, is the conditional probability of getting some data $d$ given that the
data can be modelled with some parameters $\theta$, and $\mathcal{Z}$ is the marginalised likelihood known as the
evidence. The evidence is a single number that describes how well a model fits the data. The larger the evidence, the
better the model is at fitting the data.


## The Bayesian Coherence Ratio {#BayesianCoherenceRatio}

The `BCR` is an odds (evidence ratio) comparing two hypotheses: `there's
a coherent GW in the data, plus Gaussian noise` vs `there's only
incoherent Gaussian noise or a glitch (modeled as a CBC) in each
detector` [@bcr_paper]. It's a generalized version of the `BCI`,
described in Section~\@ref(modeledSearches), in which the second
hypothesis is swapped for `there's only a glitch (modeled as a CBC) in
each detector`. The `BCR` has 2 tunable parameters ($\alpha$ and
$\beta$) that allow us to better separate signals from noise.

An example of how the `BCR` performed to distinguish glitches from
actual GW signals is shown in Figure~\@ref(fig:bcrSnr). The figure
demonstrates that while SNR was unable to distinguish the signals
(simulated/hardware/actual signals) from the background triggers, the
`BCR` was able to successfully do so.


(ref:bcrSnr) BCR versus SNR distributions from simulated signals (blue), hardware injections (green), GW events (stars) and background triggers (black), taken from @bcr_paper.

```{r bcrSnr, echo=FALSE, fig.pos= "!h", fig.asp=.7, fig.width=6, fig.align='center', out.width='75%', fig.cap='(ref:bcrSnr)', fig.scap="BCR versus SNR for O1"}
knitr::include_graphics("images/bcr_vs_snr.png")
```

Figure~\@ref(fig:bcrSnr) also demonstrates that some simulated signals and LVT1510's 
significance increases when using the BCR inplace of using the SNR as a detection 
statistic. This was the initial motivation behind our current work. 


[`cWB`]: https://gwburst.gitlab.io/
[`PyCBC`]: https://pycbc.org/
[`oLIB`]: https://arxiv.org/pdf/1511.05955.pdf
[`GstLAL`]: https://lscsoft.docs.ligo.org/gstlal/