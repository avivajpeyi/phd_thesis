\chapter{Introduction}
\label{cp.intro}


% Statistical inference in astrophysics is a fast-evolving field with a unique set of challenges. Broadly speaking, astrophysical observations allows us to probe fundamental physics in a way that would be impossible on Earth. The strong gravity around black holes and the extreme states of matter in neutron stars could not possibly be replicated in a laboratory and thus provide a unique way to understand the universe. However, the obvious trade- off is that these objects are far away and much of the most interesting data we record stems from faint, transient events. These transient events can in general be seen as a not controlled and not reproducible natural experiment, and thus require careful statistical treatment in their interpretation.
% At the time of writing only 50 confident gravitational-wave transients from the mergers of compact objects have been reported by the LIGO/Virgo Scientific Collaboration [20, 29], and only one optical counterpart was ever detected for GW170817, a merger of two neutron stars [277]. Still, these observations provide some of the best probes of extreme gravity in many aspects. Thus, we must analyse these events carefully and use optimal techniques to infer if their physical behaviour matches our models.

% Talk about detectors and data

\section{Bayesian Inference}

\subsection{parameter estimation with bayesian inference}

\subsection{sampling}


\section{Gravitational Waves}

\section{Exoplanets}