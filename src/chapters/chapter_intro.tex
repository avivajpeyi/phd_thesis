\chapter{Introduction}
\label{cp.intro}

From ancient times to the modern day, astronomy has played an important role in all cultures and eras.
Questions about the nature of the infinite cosmos and our place in it have intrigued great minds for millennia.
In researching the cosmos, astrophysicists try to comprehend the vast scale of the universe, both spatial and temporally.
Astronomers try to develop instruments that can tap into this ocean of cosmic data and make sense of the flood of information that results.

Each new drop of cosmic data obtained improves our understanding of the universe.
For example, consider the history of stellar evolution.

In 1872, astronomers began recording spectroscopic data for thousands of stars in the Henry Draper Catalogue.
Antonia Maury, a female astronomer working in the observatory, noticed that some stars in the catalog were much brighter than other stars of the same color, and sorted those brighter stars into a different category from their less-luminous brethren.

Several years later, Einjar Hertzpsrung, working on his own star classifiction project in 1905, noticed the same thing she had: some stars had very similar colors but very different brightness.
When he found that Maury had accounted for that in her catalog, he used it as a reference for his work.

Several years later, in 1913, Henry Norris Russell demonstrated the effect far more strikingly because by that time, there was more data availible. Russel plotted the absolute magnitude against spectral type for Maury's stars, nearby stars with parallaxes measured at the time, stars from the open cluster Hyades, and several moving groups.
This plot, known as an Hertzpsrung-Russel (HR) diagram, has been recreated with modern stellar data in Figure X.

HR FIGURE


 Russell noticed that the great majority of the stars in the catalogs of his time fell in a band which ran from the upper left to the lower right.

 He indicated this in his version of the diagram with lines (plotted in gray).
 We now call this region the main sequence, however at the time Russell named these the `dwarf' stars, in comparion to the stars above the lines, which he dubbed the `giant' stars.



 This taught us that
 1. there appears to be a relation between the XX and YY
 2.


 He also noticed the difference in the other two corners of the diagram: there were some stars in the upper right, but very very few stars in the lower left. Hmmmm. What could this mean?

Well, think for a moment about these corners. Stars in the upper right region are powerful and cool.

Right. These stars must be very large -- they must have a very big surface area in order to produce so much power with at such low temperatures. Hertzsprung gave the name giant to the stars with this combination of features, and we have kept it ever since. You often see the term red giant, since most of the stars are reddish. Note that this important feature of the stellar population -- that cool stars come in two sizes -- is a simple consequence of the gap between the bright and faint groups on the right-hand side of this diagram.

Both Hertzsprung and Russell noticed this gap between the luminous and feeble red stars, and decided to give names to the two groups. If the powerful red stars are called "giants", then the feeble red stars should be called .... "dwarfs."


However, Russell realized that these "red dwarfs" were really just the cool end of the main sequence of stars. As a result, he decided to extend the term dwarf to refer to ANY member of the main sequence, not just the red end.


The lower left corner: white dwarfs
At the time Hertzsprung and Russell created this diagram, they knew of very few stars which fell in its lower left corner: stars which were hot, yet feeble.

This was partially only made possible because of the telescope and data.




Fastforward to 1993.

By this time, we had newer telescopes with much better XX.
We could funnel more cosmic data from the ocean.

Ffibure (B) created with hippacros, yale, gip catalogs. These together
contain over ZZ stars.

These taught us YYY


Subgaints, supergiants, white dwarves, main sequence.



Finally, moving to 2018 + gaia. More data, new knowledge.
Recording data is also much faster.




With an inordinate amount of data being produced using modern instruments, astronomy has become more data driven than ever before.


Modern astronomy finds its basis in the streams of data delivered by ground and space-based telescopes across all wavelengths with frontiers emerging in the field of neutrino physics, high-energy astrophysics, and gravitational waves.
With the enormous amount of data funnelling through the observational centres or high-end simulations, knowledge processing and accumulation in astronomy is growing exponentially.

However, it took more than three long decades for us to cross the threshold of technology-driven science.



An accelerating amount of data is now being received to help us understand the physical universe. With the advent of the internet age and the establishment of public data archives, we have also democratised data access and its processing. This coming of age in astronomy not only revolutionized it but also led to a paradigm shift in the nature of astronomical research and its method over the turn of the last century.



Evolving from location and laboratory-intensive research to global collaborative data-driven projects, the world has come quite far.  We are in the transition into the ever-evolving new. This also gives us an opportunity to archives, assess, and communicate the becoming of data-driven astronomy from the 1980s until now. It is the historicization of such intense research material over these decades that makes us understand the intellectual movement in the field but also helps us assess emerging fields like astroinformatics.  T
he gradual shift from analogue to digital observational methods led to the shift in data-collection regimes.
Astronomers have moved from photographic, and electronic, to born-digital data, which moved the discovery away from the telescope itself to hard drives and data archives. T
his not only becomes a study in the history of technology but also an intellectual history of knowledge networks and global collaborations.



Gaia HR diagram has binaries


The most prominent feature of the giant branch is the Red
Clump (RC in Fig. 10, around GBP − GRP= 1.2, MG = 0.5 mag).
It corresponds to low-mass stars that burn helium in their core
(e.g. Girardi 2016). The colour of core-helium burning stars is
strongly dependent on metallicity and age. The more metal-rich,
the redder, which leads to this red clump feature in the local
HRD. For more metal-poor populations, these stars are bluer and
lead to the horizontal-branch (HB) fe


% Statistical inference in astrophysics is a fast-evolving field with a unique set of challenges. Broadly speaking, astrophysical observations allows us to probe fundamental physics in a way that would be impossible on Earth. The strong gravity around black holes and the extreme states of matter in neutron stars could not possibly be replicated in a laboratory and thus provide a unique way to understand the universe. However, the obvious trade- off is that these objects are far away and much of the most interesting data we record stems from faint, transient events. These transient events can in general be seen as a not controlled and not reproducible natural experiment, and thus require careful statistical treatment in their interpretation.
% At the time of writing only 50 confident gravitational-wave transients from the mergers of compact objects have been reported by the LIGO/Virgo Scientific Collaboration [20, 29], and only one optical counterpart was ever detected for GW170817, a merger of two neutron stars [277]. Still, these observations provide some of the best probes of extreme gravity in many aspects. Thus, we must analyse these events carefully and use optimal techniques to infer if their physical behaviour matches our models.

% Talk about detectors and data

\section{Bayesian Inference}

\subsection{parameter estimation with bayesian inference}

\subsection{sampling}


\section{Gravitational Waves}

\section{Exoplanets}
