\chapter{Intermediate-mass black holes}\label{apdx.imbh}

In 1932, Landu and Chandrasekhar theorized the endpoint of massive
stars that collapse into singularities as Black holes (BHs). 
In the 1970s, X-ray and optical observations from the Cyg-X1 binary star system lead to the discovery of the first stellar mass black hole. 
Shortly afterward, in 1974, kinematic measurements of stars near the center of our galaxy demonstrated the presence of a supermassive black hole of mass $10^6\ M_{\odot}$ -- Sag A*~\cite{sagA}.

These two detections firmly established the existence of stellar mass black holes $10^{2} \ M_{\odot} > M_\text{BH}$ and supermassive black holes (SMBH), $10^{5} \ M_{\odot} < M_\text{BH}$ in our universe. 
Since then, more evidence for various black holes has been acquired. 
However, until 2019, none of the bonafide black holes detected have masses between $10^{2} - 10^{5}\ M_{\odot}$. 
The elusive black holes in this mass range are known as intermediate-mass black holes (IMBHs).
This appendix dives into some of their formation scenarios, and techniques used to try and observe them.  


\section{Formation Scenarios}

To help reflect on the various formation scenarios for IMBHs,  
as suggested by \citet{miller2003formation},
one can recall the following Shakespearean quotation:

\myblockquote{Some are born great, some achieve greatness, and some have greatness thrust upon 'em. }{Shakespeare:  Twelfth Night, Act II Scene V}


This quote inadvertently summarises the popular IMBH formation scenarios: some IMBHs might form directly from the collapse of massive stars or pristine-gas galaxies~\cite{lodato2006supermassive}, some IMBH might start as stellar mass black holes that accrete enough mass to fall into the intermediate regime~\cite{coleman2002production}, and finally some IMBH might be created via repeated mergers of stellar-mass black holes~\cite{2018IMBHreview}. 
These scenarios are discussed in Section~\ref{scenarios}. 
Section~\ref{apdx.imbhlocations} provides hypotheses on where these IMBHs might be located in our universe after formation.

There are various postulated scenarios of IMBH formation. 
The scenarios discussed in this section are displayed in the schematic diagram presented in Figure~\ref{fig:imbh-sources}. 
These formation channels can be divided into two main groups based on the seeds that lead the IMBH to form: light and heavy seeds. 
For a more comprehensive list and details of IMBH formation scenarios, refer to~\cite{van2004intermediate}.


%% FIGURE
% (ref:imbh-sources) Schematic diagram of various formation scenarios of IMBHs and SMBHs (figure adapted from @mezcua2017observational).

% ```{r imbh-sources, echo=FALSE, fig.pos= "!h", fig.asp=.7, fig.width=6, fig.align='center', out.width='90%', fig.cap='(ref:imbh-sources)', fig.scap="IMBH formation channels"}
% knitr::include_graphics("images/imbh_sources.png")
% ```


\subsection{Light Seed}

The light-seed formation channels discussed in this section are
displayed in the low-mass region, $10^{1-2} \ M_{\odot}$, of
Figure~\ref{fig:imbh-sources}. These channels are:

\begin{enumerate}
\item \textbf{Accretion onto a stellar mass black hole:} BHs can accrete mass from its surrounding at the Eddington rate.
This process can increase the BHs mass. Hence, gradual accretion
onto a stellar-mass BH could lead to the formation of an IMBH.
However, the accretion rate is small, leading to growth timescales
longer than the time needed for a BH to cross a molecular
cloud, or even the lifetime of such a cloud~\cite{coleman2002production}.
\item \textbf{Population III stars:} Population III stars are hypothetical, extremely hot, massive stars (with masses $10^{2-3}\ M_{\odot}$) that form from metal-free primordial gases~\cite{bromm2013formation}. 
\citet{heger2003massive} point out that there is a `mass-gap' between $140-260 \ M_{\odot}$ a region where if a Population III star collapses, it has no remnant. 
Only if the star's mass is above $260 \ M_{\odot}$ can it form a BH of at least half of the original star's mass (and hence form an IMBH).
\item \textbf{Runaway-collisions in Low-Metallicity Clusters:} \citet{vanbeveren2009stellar} simulated the dynamics of young, dense stellar systems, including the effects of stellar wind mass loss. 
With the inclusion of stellar-wind mass loss, they were able to model
successive collisions of the most massive stars in the stellar
systems. 
These collisions, known as runaway collisions, formed very
massive stars (with mass $> 120 \ M_{\odot}$) in their
simulations. These $> 120 \ M_{\odot}$ mass stars left behind
BHs with mass $\approx 70\ M_{\odot}$, and in low metallicities,
BHs with mass $\approx 100\ M_{\odot}$ (at the edge of the IMBH
regime). These runaway-collisions IMBHs could hypothetically form in
the nuclear cluster of $2^\text{nd}$ generation stars~\cite{mezcua2017observational}.
\item \textbf{Mergers of stellar mass black holes:} In old stellar clusters ($\sim10^8$ yrs old), most of the massive objects are stellar remnants (the end stages of evolution for stars). 
For many years it had been hypothesized that BHs could form binaries in such clusters and eventually merge. In 2015, LIGO prove the existence of binary BHs (BBHs) by detecting gravitational waves generated by the merger of a BBH~\cite{abbott2016observation}. The remnant mass of this merger was $\sim63\ M_{\odot}$. Since then, LIGO has detected several more GWs from BBHs (see Figure~\ref{fig:cbc_mergers}). 
Thus, it might be possible to detect gravitational waves from the merger of a BBH that results in the formation of an IMBH, or even an IMBH-BH merger~\cite{salemi2019search}. 
The expected rates of such mergers are discussed in Section~\ref(motivations).
\end{enumerate}


The difficulties with forming IMBHs through BH mergers are that the
escape velocities of globular clusters are very low at $\sim50$km/s~\cite{morscher2013retention}. 
Thus, if another compact object passes by the binary, the binary might get ejected from the globular cluster.
Although the ejected binary will still be able to merge, the post-merger remnant will be unable to undergo a $2^{\text{nd}}$ generation merger (due to the lack of other BHs in its vicinity), leading to a smaller chance of accumulating mass via mergers.  
Similarly, if the BBH does manage to merge inside the globular cluster, gravitational waves from the coalescence of the BBH  will carry away linear momentum, causing the remnant post-merger BH to get a ``recoil kick''. 
This ``kick'' effect could lead to the ejection of the post-merger BH from the globular cluster~\cite{favata2004black}.

It is important to note that if one of the BH's has an initial mass $>50 \ M_{\odot}$ while the other has a comparatively smaller mass, the larger BH can undergo successive mergers and stay inside the cluster due to inertia~\cite{coleman2002production}.

%%%% FIGURE KICK


\subsection{Heavy Seed}

The heavy-seed formation channels discussed in this section are
displayed in the high-mass region, $10^{5-8} \ M_{\odot}$, of
Figure~\ref{fig:imbh-sources}. 
The two heavy-seed channels in the Figure both involve the first metal-free protogalaxies.

In the first channel, if fragmentation is prevented, rapidly inflowing dense gas of the protogalaxy can form a supermassive star of  $\sim10^{5} \ M_{\odot}$~\cite{regan2014direct}. 
The collapse of such a large star can form a black hole of comparable mass~\cite{regan2014direct}.

The second channel considers the mergers of metal-free protogalaxies.
Mergers of metal-free protogalaxies can form massive compact nuclear
gas disks, with a supermassive star at its center. This star can
collapse to form an IMBH or even an SMBH of  $\sim10^{8} \
M_{\odot}$~\cite{mezcua2017observational}.


\section{SMBH formation via IMBHs}

We have evidence that SMBHs formed at $z\sim7$ or 0.8 Gyr after the
formation of the Universe~\cite{mezcua2017observational}. However, we are
unsure of how these SMBHs form. A popular notion was that stellar mass
BHs that formed soon after the Big Bang accreted enough mass to form
into these SMBHs. However, if stellar mass BHs obey the Eddington
Limit, 0.8 Gyr is too short a time frame for stellar mass BHs to
accrete enough mass to turn into SMBHs. This is where IMBHs come in --
instead of stellar mass BH, there might have been IMBH in the early
universe as the seeds of SMBHs. Hence several formation scenarios of
SMBHs propose that they pass through a life stage in which they are
IMBHs~\cite{mezcua2017observational}. Additionally, the IMBHs which do not
turn into SMBHs should still be present as leftover IMBHs.


Based on these concepts, we can try to speculate where we might find
IMBHs.


\section{Current Locations of IMBHs}\label{apdx.imbhlocations}

According to \citet{mezcua2017observational}, leftover IMBHs that did not turn
into SMBHs should be found in dwarf galaxies, globular clusters, young
star clusters, and finally, some might be found orphaned from globular
clusters.

One possible formation channel for IMBHs we discussed were from multiple
BH mergers. If the IMBH that forms from the merger manages to stay
inside the globular cluster, IMBH might form an active galactic nuclei
(AGN). If this IMBH's host galaxy is close to other galaxies hosting
IMBHs, then the galaxies might merge, leading to an IMBH merger.
Subsequent IMBH mergers like this would turn the IMBH into an SMBH.
However, if the galaxy is an isolated dwarf galaxy, then the IMBH might
be able to remain as an IMBH. Hence, IMBHs might be found at the center
of globular clusters, or in isolated dwarf galaxies.

Another formation channel involved runaway collisions of stars with low
mentalities. If such runaway collisions can occur in regions with no
stellar winds, then IMBHs might form in young star clusters where these
collisions take place.

% Some examples of candidate galaxies with IMBHs can be seen in
% Figure~\@ref(fig:example-locations).

% (ref:example-locations) The globular cluster 47~Tucanae, the dwarf galaxy NGC~4395 and the spiral galaxy ESO~243-49 are all host-galaxies for candidate IMBHs.

% ```{r example-locations, echo=FALSE, out.width='30%', fig.show='hold', fig.align='center', fig.cap='(ref:example-locations)', fig.scap="Candidate IMBH host-galaxies"}
% knitr::include_graphics(c("images/47tuc.png", "images/ngc4395.png", "images/eso243.png"))
% ```



\section{Motivations for finding}

Even though the existence of IMBHs is debatable, these objects if
discovered can play a pivotal role in understanding various
astrophysical processes. If an IMBH were to be observed our
understanding of the formation and evolution of SMBHs, as well as galaxy
evolution modeling and theories of the strong-field regime of gravity, could
dramatically change~\cite{mezcua2017observational}. We might also understand
more about tidal disruption of stars~\cite{IMBHcauseTidalDisruption}, and
the quenching of star formation in early galaxies~\cite{kim2011galaxy}.
These, along with some other motivations for detecting IMBHs are
discussed in this section.


\paragraph{SMBH and galaxy evolution:} 
Detecting IMBHs will be pivotal in understanding SMBH and galaxy
evolution. As mentioned earlier, most formation scenarios of SMBHs
propose that they pass through a life stage in which they are IMBHs
~\cite{mezcua2017observational}. Understanding the formation of IMBHs and SMBHs will in turn yield
insights to the early structure of the Universe. For example, the
number of present-day IMBHs can elucidate how seed BHs formed in the
early Universe: either from the death of Pop III stars, the direct
collapse of pristine gas, or by stellar mergers in dense stellar
clusters (each have different populations for IMBHs).

\paragraph{Test of GR:} The detection of the gravitational capture of a stellar-mass compact object by an IMBH will allow us to test gravity in the strong regime ~\cite{GWimbhProspects}.

\paragraph{Source of gravitational waves:} 
\citet{abadie2010ratePredictions} noted that the portions of GWs of certain
IMBH binaries (IMBHBs) and IMBH-BH binaries should be detectable in
the advanced LIGO and VIRGO interferometers. Simulations by \citet{fregeau2006imbhbRatePrediction} have demonstrated
that IMBHBs might form during runaway collision scenarios in young
star clusters. According to the simulations, the recently formed
IMBHB quickly tighten into a common binary shortly after their birth
and quickly inspiral and merge. The rate of such IMBHB mergers
detectable by aLIGO and VIRGO is theorized from this simulations to
be $\sim10\ \text{yr}^{-1}$. \citet{abadie2010ratePredictions} points
out another formation channel for IMBHBs: from after the
collision of two globular clusters (each with an IMBH). However,
no rates for such IMBHB mergers exists. In a similar vein, simulations of interactions inside globular
clusters demonstrate that IMBH-BH pairs with low mass-ratios may
form~\cite{mandel2008rates}. Such binary systems might undergo
tightening through various three body interactions, and eventually
merge. Due to the extreme mass ratio of this scenario, the scenerio
is named an intermediate mass-ratio coalescence, IMRAC. According to \citet{mandel2008rates} such events can occur anywhere from 0-10 times a year. The rates for such events are tabulated in Table~\ref{tab:ligo-rates}.


%% TABLE 
% (ref:imbhRatePrediction) @fregeau2006imbhbRatePrediction

% (ref:mandel2008rates) @mandel2008rates

% (ref:rodriguez2015bbhRatePredictions) @rodriguez2015bbhRatePredictions

% ```{r ligo-rates, tidy=FALSE, echo=FALSE}

% data <- read.csv('data/imbh_rates.csv',  stringsAsFactors = FALSE)

% knitr::kable(
%   head(data), booktabs = TRUE, escape =F,
%   col.names = c("Black Holes","Merger Type","Event Rate $r$ [1/yr]","Reference"),
%   caption = 'A table describing the different detection rate estimates in the aLIGO band.',
%   caption.short = "High-mass detection rates in aLIGO" 
% )
% ```


\section{Observational Methods}

To make a definitive detection of an IMBH, a clear measurement of the
IMBH's mass is required~\cite{koliopanos2018intermediate}.  There
are two broad categories of observational methods employed to detect
IMBHs, direct and indirect observational methods.

\subsection{Direct observational method}

Some of the direct methods used to measure SMBH masses, such as dynamic
kinematic measurements and reverberation mapping, have been employed
to measure the masses of various IMBH candidates. However, none of these
studies have proved conclusive as the sphere of influence of IMBHs are
too small to be resolved.

Dynamic kinematic measurements of individual sources orbiting a BH, such
as kinematic measurements of orbits around Sgr A*~\cite{sagA}, is the most reliable method of measuring BH
mass. \citet{47TucHostIMBH} studied the motion of 19-millisecond pulsars in the
globular cluster 47-Tucanae and described their motion using an IMBH
with mass ${\sim2300\ M_{\odot}}$. However, others were able to
describe the motions without an IMBH~\cite{freire2017long}.


Reverberation mapping is a procedure where time-delays of radiation from
a BH and its orbiting gas can be used to infer mass of the BH. 
\citet{ncg4395} analysed the timing delays from an IMBH candidate
in Dwarf galaxy NGC-4395 and estimated that a IMBH with mass
${\sim3.6\times10^5\ M_{\odot}}$ was the source of the time-delay
radiation. However, because we cannot discount the possibility that the
radiation from the source could be anisotropic or non-linear (leading to
overestimating the true source mass), reverberation mapping by
itself is inadequate.

Another drawback of direct methods is that they cannot probe beyond
only the most nearby galaxies. To be able to study even the smaller, less
luminous, and further away IMBHs, indirect methods of observation can be
used.

\subsection{Indirect Observational Methods}

Several indirect observational methods have been employed
to search for IMBHs (see \citet{mezcua2017observational} for a detailed
study on the various methods and their results). Here, we discuss
looking for IMBHs by studying low-luminosity AGNs, ultraluminous x-ray
sources, and gravitational waves.

\subsection{Studying x-ray emission}
AGNs create unique optical spectra due to the ionization of gas,
surrounding the BH, from x-rays released from accretion discs
\citet{elvis2000structure}. When the emission lines from this optical spectra are
studied alongside the broad-line emission lines from the central BH of
the AGN, a virial estimate of the BH mass can be calculated
~\cite{IMBHinLLAGN}. Although several groups have claimed to classify IMBHs
by studying such AGNs, the results have large uncertainties and are
contradictory.

In a similar vein there have been numerous studies of x-ray emissions
(from off-nuclear extragalactic sources) exceeding the Eddington limit
for stellar mass BH~\cite{kaaret2017ultraluminous}. These sources, known as
ultraluminous X-ray sources (ULXs) make a large IMBH candidate set
(under the assumption that the emission is isotropic). One such IMBH
candidate labeled HLX-1 located in the galaxy ESO~243-49 is considered
to have a mass of $\text{M}\sim5000\ M_{\odot}$ due to a luminosity of
$L_x\sim10^{42}\ \text{erg s}^{-1}$~\cite{hlx1}.

\subsection{Studying GWs}
Another pathway to detecting an IMBH is with gravitational waves emitted
from compact binary black hole systems. As the frequency at which
gravitational waves are emitted is inversely proportional to the chirp
mass of the system, higher mass systems correspond to lower frequency
gravitational waves~\cite{abbott2016properties}. Currently, there are three
ground-based detectors (LIGO-Hanford, LIGO-Livingston and Virgo) that
are operational and have successfully made GW detections of binary BH
(BBH) and binary neutron star (BNS) mergers~\cite{abbott2019gwtc}. 
These detectors are not
sensitive to supermassive ($\sim10^9\ M_{\odot}$) and massive
($\sim10^{4-7}\ M_{\odot}$ binary systems. However, as discussed
in Section~\ref{motivations}, these detectors might be able to detect
GWs from an IMBH system with a total mass less than $500\
M_{\odot}$~\cite{salemi2019search}.

In July 2019, the LIGO-Virgo scientific collaboration released the
outcomes of an IMBH search they conducted on their first and second
observing runs. Although their work did assist in constraining IMBHB
merger rates, they could not detect an IMBH binary merger~\cite{salemi2019search}.

A potential reason they provided for why no IMBHBs were found was
because of detector noise. As IMBHs have high mass, the IMBH GW signals
have a low frequency and a duration lasting less than a second~\cite{salemi2019search}. This corresponds to a short-duration trigger in the
LIGO data. A trigger is a section of LIGO data identified by data
searching pipelines to potentially harbour some GW signal data (more on
triggers in Section~\ref{searches} where the search pipelines are
discussed). In the ground based GW detectors, there are many spurious
instrumental noise events, that fool the pipelines into considering them
as triggers. These instrumental-triggers are called glitches. Luckily,
unlike triggers from astrophysical sources, glitches do not often occur
at the same time or with the same morphology across the various GW
detectors -- they are often incoherent. However, as glitches appear very
similar to expected short duration IMBH signals, it can be difficult
distinguishing the two. 

\citet{salemi2019search} stated that increases of
the GW detector network sensitivity and search techniques in future
runs might result in a detection of GWs from an IMBH. The following
Section~\ref{searches} discusses some of these searches, and introduces
a novel search for IMBHs that can better discern between IMBH signals
and glitches.

