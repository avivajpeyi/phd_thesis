\chapter{Summary and Conclusions}
\label{cp.conc}
In this thesis, I have XX.
I first provided a thorough overview of computational Bayesian analysis for gravitational-wave inference, introducing the software libraries underlying the bulk of the analysis in this thesis.
I then demonstrated GW151226...
These results tell us about astrophysics

In the last chapter of this thesis, I have illustrated a proof-of-concept design for the kind of science that may be facilitated by gravitational waves in the future, when conclusive measurements of mass and spin, and eccentricity will identify the AGN mergers in the population...

In the remainder of this thesis, I conclude with some thoughts about how this work should be improved and extended in the coming years.

\section{Infereing parameters of }

TESS, BCR, Pbilby, Deep Followup tell us about systems

\section{Population}
TESS AGN will tell us about populations





With an inordinate amount of data being produced using modern instruments, astronomy has become more data driven than ever before.


Modern astronomy finds its basis in the streams of data delivered by ground and space-based telescopes across all wavelengths with frontiers emerging in the field of neutrino physics, high-energy astrophysics, and gravitational waves.
With the enormous amount of data funnelling through the observational centres or high-end simulations, knowledge processing and accumulation in astronomy is growing exponentially.


An accelerating amount of data is now being received to help us understand the physical universe. With the advent of the internet age and the establishment of public data archives, we have also democratised data access and its processing. This coming of age in astronomy not only revolutionized it but also led to a paradigm shift in the nature of astronomical research and its method over the turn of the last century.



Evolving from location and laboratory-intensive research to global collaborative data-driven projects, the world has come quite far.  We are in the transition into the ever-evolving new. This also gives us an opportunity to archives, assess, and communicate the becoming of data-driven astronomy from the 1980s until now. It is the historicization of such intense research material over these decades that makes us understand the intellectual movement in the field but also helps us assess emerging fields like astroinformatics.  T
he gradual shift from analogue to digital observational methods led to the shift in data-collection regimes.
Astronomers have moved from photographic, and electronic, to born-digital data, which moved the discovery away from the telescope itself to hard drives and data archives. T
his not only becomes a study in the history of technology but also an intellectual history of knowledge networks and global collaborations.


