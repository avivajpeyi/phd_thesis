% https://www.monash.edu/graduate-research/examination/publication
% https://www.monash.edu/rlo/graduate-research-writing/write-the-thesis
% https://www.monash.edu/rlo/graduate-research-writing/write-the-thesis/writing-the-thesis-chapters/structuring-a-long-text
\abstract{
\addtocontents{toc}{}  % Add a gap in the Contents, for aesthetics

In the past, astronomers like Ptolemy and Galileo were constrained by their inability to reliably make and record observations.
However, nowadays, astronomy is experiencing an explosion of data and observations.
Modern observatories produce petabytes of data, and the simulations to model these observations push computers to their limits.
Processing and synthesising this data is a major challenge in astrophysics.
Thanks to statistical approaches, we can make inferences from a sample of simulations to draw astrophysical conclusions from the data we have.
This dissertation shows how a particular statistical technique, Bayesian Inference, can be utilised to disclose previously unknown insights about the cosmos.
Specifically, it demonstrates how Bayesian approaches can be applied to the analysis of astronomical time series data, such as gravitational-wave data or light-curve with transits, to characterise events like the merger of binary black holes and the transit of exoplanets.
Furthermore, this thesis shows the efficacy of Bayesian model selection by presenting a follow-up search for intermediate-mass black holes in LIGO data.
Finally, the thesis concludes with a discussion of Bayesian approaches for investigating ensemble properties of populations, using a case study of black hole spins in active galactic nuclei as an illustration.

}
