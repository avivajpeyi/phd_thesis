% https://www.monash.edu/graduate-research/examination/publication
% https://www.monash.edu/rlo/graduate-research-writing/write-the-thesis
% https://www.monash.edu/rlo/graduate-research-writing/write-the-thesis/writing-the-thesis-chapters/structuring-a-long-text
\abstract{
\addtocontents{toc}{}  % Add a gap in the Contents, for aesthetics

The abstract should outline the main approach and findings of the thesis and must not be more than 500 words.

The Thesis Abstract is written here (and usually kept to just this page). The page is kept centered vertically so can expand into the blank space above the title too\ldots

% In this thesis, I explore how Bayesian inference is used in the analysis of gravitational waves, and in the analysis of astrophysical time series more broadly. I present the Bilby project, a collaborative software package that has now become a standard analysis tool for gravitational-wave signals. One specific application of Bayesian inference is the search for nonlinear gravitational-wave memory, a hereditary effect in which gravitational-wave strain builds up during an accompanying burst and remains at a constant value different from zero after the wave has passed. This thesis presents the results of measuring mem- ory in the first and second LIGO/Virgo gravitational-wave transient catalogue. This catalogue contains 50 events that have been associated with the coalescence of compact binary objects, with most of them being black holes. Combining the results from these events I find that the hypothesis that memory is present in the catalogue is very slightly favoured but not statistically significant. While memory will not be detectable in a single gravitational-wave signal with current generation detectors, we provide the infrastruc- ture with which to do it in O(2000) events, a milestone likely reached by the end of the decade.
% The generality and versatility of Bilby allows for the easy deployment of inference meth- ods in wide contexts such as in astrophysical x-ray time series. I present the discovery of a new bias in the search for quasi-periodic oscillations using frequency-domain meth- ods that arises in non-stationary time series. Not accounting for this bias can lead to a strong overestimation of the significance of quasi-periodic oscillations, especially when periodograms and Whittle likelihoods are used. I show that one way of mitigating this ef- fect is to employ Gaussian process modelling. Gaussian processes are in principle superior as they can model deterministic trends, non-stationary behaviour, and heteroscedastic data, but are more computationally expensive and involve more complex modelling pro- cedures. Nevertheless, I show that using a narrow class of kernel functions Gaussian processes are a viable method for finding and characterising quasi-periodic oscillations.




In the past, astronomers like Ptolemy and Galileo were limited by their ability to make and record observations.
However, nowadays, astronomy is experiencing an explosion of data and observations.
Modern observatories produce petabytes of data, and the simulations to model these observations push computers to their limits.
One core struggle in astrophysics is processing and synthesising this data.
Statistical methods make it possible to use a subset of simulations to yield astrophysical inferences about the observed data.
This thesis demonstrates how a specific statistical method, Bayesian Inference, can be used to discover novel information about the universe.
This thesis describes how astrophysical time-series data, such as gravitational-wave data or light-curve with transits, can be analysed with Bayesian methods to characterise merging binary black holes and transiting exoplanets.
This thesis exhibits the power of Bayesian model selection by presenting a follow-up search for intermediate-mass black holes in LIGO data.
Finally, the thesis explains Bayesian methods to investigate ensemble properties of populations by demonstrating a study of black hole spins in active galactic nuclei.



}
