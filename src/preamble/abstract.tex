% https://www.monash.edu/graduate-research/examination/publication
% https://www.monash.edu/rlo/graduate-research-writing/write-the-thesis
% https://www.monash.edu/rlo/graduate-research-writing/write-the-thesis/writing-the-thesis-chapters/structuring-a-long-text
\abstract{
\addtocontents{toc}{}  % Add a gap in the Contents, for aesthetics


In the past, astronomers like Ptolemy and Galileo were limited by their ability to make and record observations.
However, nowadays, astronomy is experiencing an explosion of data and observations.
Modern observatories produce petabytes of data, and the simulations to model these observations push computers to their limits.
One core struggle in astrophysics is processing and synthesising this data.
Statistical methods make it possible to use a subset of simulations to yield astrophysical inferences about the observed data.
This thesis demonstrates how a specific statistical method, Bayesian Inference, can be used to discover novel information about the universe.
This thesis describes how astrophysical time-series data, such as gravitational-wave data or light-curve with transits, can be analysed with Bayesian methods to characterise merging binary black holes and transiting exoplanets.
This thesis exhibits the power of Bayesian model selection by presenting a follow-up search for intermediate-mass black holes in LIGO data.
Finally, the thesis explains Bayesian methods to investigate ensemble properties of populations by demonstrating a study of black hole spins in active galactic nuclei.



}
